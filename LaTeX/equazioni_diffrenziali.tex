\documentclass[20pt]{article}
\parindent 0px

\begin{document}

    \begin{center}
        \begin{Large}
            EQUAZIONI DIFFERENZIALI\\[10pt]
        \end{Large}
    \end{center}

    \underline{Definizione}:\\[5pt]
    Un equazione differenziale è un qualcosa che ha per incognita \underline{$y=f(x)$} e che contiene almeno una delle sue derivate.
    $$y^{'}=5x \; (I \; oridine)$$

    L'ordine di un equazione differenziale è l'ordine massimo     delle derivate che \. compaiono \. nell'equazione.
    \\[5pt]
    Soluzione generale o integrale generale: Soluzione di un         'equazione generale
    \\[17pt]

    \begin{center}
    EQUAZIONE DIFFERENZIALE DI I ORDINE\\[10pt]
    \end{center}
    Forma generale:
    $$y^{'}=f(x)$$
    Esempio 1\\[10pt]
    $y^{'}-x^2+x-2=0$\\
    $y^{'} = x^2-x+2$\\
    $y=\int x^2-x+2 dx = \frac{x^3}{3}-\frac{x^2}{2}+2x+c$\\[5pt]
    
    Esempio 2:\\[10pt]
    \begin{large}
        \begin{math}
          \left\{
            \begin{array}{l}
              \frac{y^{'}-x}{x^2+1} = \frac{3}{x}\\
              f(1)=0 \; (condizione \; iniziale)
            \end{array}
          \right.
        \end{math}
    \end{large}\\[5pt]
    
    \begin{large}
        $\frac{y^{'}-x}{x^2+1} = \frac{3}{x} \rightarrow$ 
        $y^{'}-x = 3 \frac{x^2+1}{x} \rightarrow$
        $y^{'}=\frac{3(x^2+1)+x^2}{x} \rightarrow$
        $y^{'}=\frac{4x^2+3}{x} $\\[5pt]
        $y=\int{\frac{4x^2+3}{x} dx} \; = $
        $\int{4x}dx + \int{\frac{3}{x} dx} \; = $
        $2x^2+3 \ln{|x|}+c$\\
    \end{large}
    
    Cerchiamo \underline{integrale particolare} applicando la condizione iniziale $f(1)=0$\\
    
    \begin{large}
        $y = 2x^2 + 3 \ln{|x|} + c \rightarrow$
        $ 0 = 2 + c$\\
    \end{large}    
    
    Soluzione particolare finale:
    \begin{large}
        $$y=2x^2 + 3 \ln{|x|} -2 $$
    \end{large}
    \\ [24pt]
    \begin{center}
        EQUAZIONI DIFFERENZIALI A VARIABILI SEPARABILI
    \end{center}
    
    Forma generale:\\
    $$y^{'} = g(x) \; h(y)$$
    $$\frac{dy}{dx} \; = \; g(x) \; h(y)$$
    $$\int{\frac{1}{h(y)} dy} \; = \; \int{g(x)dx}$$
    $g(x) \rightarrow \;$funzione di x\\
    $f(x) \rightarrow \;$funzione di y\\[5pt]
    
    \underline{Definizione}: \\
    A first order differential equation can be solved by separation of variables when it can be written as $y^{'} = g(x) \; h(y)$, where g and h are continous functions.\\
    
    Esempio:\\
    
    $y^{'} \; = 2x (y-1)^2 \rightarrow$
    $\frac{dy}{dx} \; = \; 2x (y-1)^2 \rightarrow$
    $\frac{1}{(y-1)^2}dy \; = \; 2x\;dx$\\
    $\int{\frac{1}{(y-1)^2}} dy \; = \; \int{2x} \; dx \rightarrow$
    $-\frac{1}{y-1} = x^2+c $
    $$y \; = \; \frac{1}{-x^2+c}+1$$
    \\[5pt]

    \begin{center}
        EQUAZIONI LINEARI DI PRIMO ORDINE
    \end{center}
    
    Forma generale: 
    $$y^{'} \; = \; a(x)y+b(x)$$
    
    I CASO: b(x) = 0 LINEARE OMOGENEA
    $$y^{'} = a(x)y \rightarrow a \; variabili \; separabili$$\\
    
    II CASO: b(x) != 0  LINEARE NON OMOGENEA\\[5pt]
    $y^{'} = -\frac{y}{x}+x$\\
    
    1) Risolvo l'omogenea associata\\
    
    $y^{'} = -\frac{y}{x} \rightarrow$
    $ \frac{dy}{dx} = -\frac{1}{x}y \rightarrow $
    $\int{\frac{1}{y}}dy \; = \; \int{-\frac{1}{y} dx \rightarrow}$
    $\ln{|y|} \; = \; -\ln{|x|}+c$\\
    
    2) $y=k(x) \; \frac{1}{x} \rightarrow$
    $k(x) \; \frac{1}{x} \; + \; k(x)(-\frac{1}{x^2}) \; = $
    $\frac{-k(x)\frac{1}{x}}{x}$
    
\end{document}