\documentclass[11pt]{article}
\parindent 0px

\begin{document}

TEOREMA DELLA MEDIA INTEGRALE\\
Definizione\\
Sia $f(x)$una funzione continua in $[a, b]$ allora $\exists \; z\in [a, b]\; t.c. \;f(z) = \frac{\int_a^b f(x) dx}{b-a}$, cioè ${\int_a^b f(x)dx = f(z)(b-a)}$
\\[10pt]
Dimostrazione\\
Poiché per ipotesi, $f$ è continua in $[a, b]$, per il teorema di weirstrass essa assume max M e min m $m \le f(x) \le M \; \forall x \in [a, b]$ 
\\[10pt]
Se considero le funzioni $y=m$, $y=f(x)$, $y=M$, per proprietà 4 dei integrali definiti:
$$ \int_a^b m \; dx \le \int_a^b f(x) dx \le \int_a^b \; M dx$$
\\[10pt]
Applicando la proprietà 5 dei integrali definiti si ottiene:
$$ m(b-a) \le \int_a^b f(x) dx \le M(b-a)$$
\\[10pt]
Dividendo tutti i membri per (b-a) si ottiene:
$$m \le \frac{\int_a^b f(x) dx}{b-a} \le M $$
\\[10pt]
Per il teorema dei valori intermedi, $f(x)$ assume tutti i valori compresi tra m e M $=> \exists \; z \; \in [a, b] \; t.c.$ 
$$f(z) = \frac{\int_a^b f(x) dx}{b-a}$$
\end{document}